\documentclass[12pt,a4paper,notitlepage]{article}
\usepackage[portuguese]{babel}
\usepackage[margin=3cm]{geometry}
\usepackage[utf8]{inputenc}
\usepackage[T1]{fontenc}
\usepackage{secdot}
\usepackage{pgfplots}
\usepackage[noend,linesnumbered,ruled]{algorithm2e}
\renewcommand{\O}[1]{$\mathcal{O}(#1)$}
\SetKwFunction{findver}{Find-Fundamental-Vertices}
\SetKwProg{Function}{function}{}{}
\SetKwInOut{Output}{Output}
\SetKwFunction{init}{initialize-graph}
\SetKwComment{Comment}{$\triangleright$\ }{}
\pgfplotsset{compat=1.12}
\begin{document}
\title{\textbf{Análise e Síntese de Algoritmos} \\\large 2º Projeto}
\date{}
\author{Tomás Cunha, nº 81201, Grupo 15}
\maketitle
\section{Introdução}
Este projeto tem como objetivo encontrar um ponto de encontro entre várias filiais de forma a minimizar o custo total das rotas, se existir.
O problema pode ser reduzido a encontrar a menor soma dos custos dos caminhos mais curtos de todas as filiais para cada localidade, representando os caminhos como arestas de um grafo e os vértices como as localidades.
Na resolução do problema utilizei a descrição do algoritmo de Johnson e da estutura de dados Min-Heap disponíveis no livro \emph{Introduction to Algorithms}\cite{algs3ed}.

\section{Descrição da solução}
A solução encontrada consiste em realizar uma variação do algoritmo Johnson tomando como vértices de fonte todas as filiais.
Em vez de guardar todos os caminhos mais curtos numa matriz, é apenas guardada a soma dos caminhos até cada localidade num vetor, reduzindo o espaço ocupado.
No final, este vetor é percorrido para encontrar a soma mínima. 
Após encontrar o ponto de encontro correto, é calculado o grafo transposto do original e realiza-se o algoritmo Dijkstra a partir do ponto de encontro, de forma a obter os custos individuais dos caminhos de cada filial até ao ponto de encontro.
\\
\\
O algoritmo pode ser representado em pseudocódigo da seguinte forma:

\begin{thebibliography}{9}
		\bibitem{algs3ed}
				Thomas H. Cormen,
				Charles E. Leiserson,
				Ronald L. Rivest,
				Clifford Stein,
				\emph{Introduction to Algorithms},
				3rd Edition,
				September 2009
\end{thebibliography}

\end{document}
