\documentclass[12pt,a4paper,notitlepage]{article}
\usepackage[portuguese]{babel}
\usepackage[margin=3cm]{geometry}
\usepackage[utf8]{inputenc}
\usepackage[T1]{fontenc}
\usepackage{secdot}
\usepackage{pgfplots}
\usepackage[noend,linesnumbered,ruled]{algorithm2e}
\renewcommand{\O}[1]{$\mathcal{O}(#1)$}
\SetKwProg{Function}{function}{}{}
\SetKwInOut{Output}{Output}
\SetKwFunction{init}{initialize-graph}
\SetKwFunction{heapify}{min-heapify} % FIXME
\SetKwFunction{extractmin}{heap-extract-min} % FIXME
\SetKwFunction{decreasekey}{heap-decrease-key} % FIXME
\SetKwFunction{result}{get-result}
\SetKwFunction{bellmanford}{bellman-ford}
\SetKwFunction{dijkstra}{dijkstra}
\SetKwFunction{h}{h}
\SetKwFunction{weight}{w}
\SetKwFunction{weightprime}{w'}
\SetKwFunction{min}{min}
\SetKwFunction{Print}{print}
\SetKwFunction{shortestpath}{$\delta$}
\SetKwFunction{shortestpathprime}{$\delta$'}
\SetKwComment{Comment}{$\triangleright$\ }{}
\pgfplotsset{compat=1.12}
\begin{document}
\title{\textbf{Análise e Síntese de Algoritmos} \\\large 2º Projeto}
\date{}
\author{Tomás Cunha, nº 81201, Grupo 15}
\maketitle
\section{Introdução}
Este projeto tem como objetivo encontrar um ponto de encontro entre várias filiais de forma a minimizar o custo total das rotas, se existir.
O problema pode ser reduzido a encontrar a menor soma dos custos dos caminhos mais curtos de todas as filiais para cada localidade, representando os caminhos como arestas de um grafo e os vértices como as localidades.
Na resolução do problema utilizei a descrição do algoritmo de Johnson e da estutura de dados Min-Heap disponíveis no livro \emph{Introduction to Algorithms}\cite{algs3ed}.

\section{Descrição da solução}
A solução encontrada consiste em realizar uma variação do algoritmo Johnson tomando como vértices de fonte todas as filiais.
Em vez de guardar todos os caminhos mais curtos numa matriz, é apenas guardada a soma dos caminhos até cada localidade num vetor, reduzindo o espaço ocupado.
No final, este vetor é percorrido para encontrar a soma mínima. 
Após encontrar o ponto de encontro correto, é calculado o grafo transposto do original e realiza-se o algoritmo Dijkstra a partir do ponto de encontro, de forma a obter os custos individuais dos caminhos de cada filial até ao ponto de encontro.
\\
\\
O algoritmo pode ser representado em pseudocódigo da seguinte forma (os algoritmos de Dijkstra e Bellman-Ford são omitidos uma vez que não foram alterados em relação aos originais):
\\
\\
\begin{algorithm}
    Let G $\leftarrow$ Grafo formado a partir do input\;
    Let F $\leftarrow$ Vértices de G correspondentes às filiais\;
    Let \weight{u,v} $\leftarrow$ Função de pesos que devolve a perda entre $u$ e $v$\;
    meeting-place, total-loss, d $\leftarrow$ \result{G, F, w}\;
    \eIf{meeting-place $=$ $\emptyset$}{
        \textbf{Output} ``N''\;
    }
    {
        \textbf{Output} meeting-place, total-loss\;
        \ForEach{v $\in$ F}{
            \textbf{Output} d[v]\;
        }
    }
    \caption{Função principal}
\end{algorithm}
\\
\\
\begin{algorithm}[H]
    \Function{\result{G, F, w}}{
        \ForEach{f $\in$ F} {
            reachable[f] $\leftarrow$ \textbf{true}\;
            sum[v] $\leftarrow$ 0\;
        }
        G' $\leftarrow$ G $\cup$ s\;
        d[v] $\leftarrow$ 0 $\forall$ v $\in$ V[G]\;
        \bellmanford{G', s, $\weight$}\;
        \h{v} = \shortestpath{s, v} calculado pelo $\bellmanford$  $\forall$ v $\in$ V[G]\; 
        \weightprime{u, v} = \weight{u, v} + \h{u} $-$ \h{v} $\forall$ $(u,v)$ $\in$ E[G]\; 
        \ForEach{u $in$ F}{
            \dijkstra{G, u, \weightprime}\;
            \ForEach{v $\in$ \shortestpathprime{u, v} calculados por $\dijkstra$}{
                \If{\shortestpathprime{u, v} = $\infty$}{
                    reachable[v] $\leftarrow$ \textbf{false}\;
                }
                sum[v] $\leftarrow$ sum[v] + \shortestpathprime{u,v} $+$ \h{v} $-$ \h{u}\;
            }
        }
        meeting-place $\leftarrow \emptyset$\;
        total-loss $\leftarrow \infty$\;
        \ForEach{s $\in$ sum}{
            \If{s < total-loss} {
                meeting-place $\leftarrow$ index[s]\;
                total-loss $\leftarrow$ s\;
            }
        }
        \If{meeting-place $\neq$ $\emptyset$}{
            d $\leftarrow$ \dijkstra{$G^{T}$, meeting-place, $\weightprime$}\;
            \ForEach{v $\in$ F}{
                d[v] $\leftarrow$ d[v] + \h{meeting-place} $-$ \h{v};
            }
        }
        \Return{meeting-place, total-loss, d}\;
    }
    \caption{Descobrir o ponto de encontro, se existir, e as distâncias das filiais a este}
\end{algorithm}
\pagebreak
\section{Análise Teórica}
A correção dos algoritmos de Johnson, Dijkstra e Bellman-Ford já foi provada.
A perda total será necessariamente dada pela soma dos caminhos mais curtos a partir de cada filial até ao ponto de encontro, logo encontrar a soma mínima irá encontrar o ponto de encontro certo.
A realização do algoritmo de Dijkstra a partir do ponto de encontro no grafo transposto irá dar os caminhos mais curtos a partir das filiais no grafo original até esse ponto de encontro: se em algum ponto a distância fosse diferente, teria obrigatoriamente de ser escolhido esse arco no caminho do grafo original, uma vez que os arcos são exatamente os mesmos.
\\
\\
Os dados do enunciado garantem que o número de filiais é sempre muito menor do que o número de localidades.
Seja \emph{F} o número de filiais, \emph{V} o número de vértices e \emph{E} o número de arestas.
Analisemos o pseudocódigo da função \textbf{get-result}, correspondente ao \emph{Algorithm 2} da secção anterior para obter a sua complexidade:

As linhas 5, 8--9, 17, 18 e 22 são executadas em $\Theta(1)$.
As linhas 2--4 têm complexidade $\Theta(F)$.
A linha 6 é executada em $\Theta(V)$.
A linha 7 executa o algoritmo de \emph{Bellman-Ford}, que tem complexidade \O{VE}.
Há um ciclo nas linhas 10--15 que é executado \emph{F} vezes.
Na linha 11, o algoritmo de \emph{Dijkstra} é executado, logo a sua complexidade é \O{(V+E)\log{V}}.
Nas linhas 12--15, há um ciclo que percorre os custos dos caminhos mais curtos para todos os vértices do grafo dados pelo \emph{Dijkstra}, tendo complexidade $\Theta(V)$.
Assim, as linhas 10--15 têm complexidade \O{$F((V+E)$\log{V}$+V)$}, mas uma vez que \O{V} é majorado por \O{(V+E)\log{V}}, é simplesmente \O{F$(V+E)$\log{V}}.
A linha 16 tem complexidade $\Theta(V)$.
A linha 19 tem complexidade \O{$(V+E)$\log{V}}, na execução do algoritmo de Dijkstra, e \O{V+E} na construção do grafo transposto. Uma vez que \O{V+E} é majorado por \O{$(V+E)$\log{V}}, tem complexidade \O{$(V+E)$\log{V}}.
As linhas 20--21 têm complexidade $\Theta(F)$.

É fácil observar que a complexidade das linhas 10--15 majora a complexidade de todas as restantes linhas, portanto o algoritmo tem complexidade \O{$F(V+E)$\log{V}}.
\\
\\
Analisando o pseudocódigo da função principal, observa-se que as linhas 1--3 são realizadas em \O{V+E+F}, a linha 4 é realizada em \O{$F(V+E)$\log{V}} (como estudado no parágrafo anterior), as linhas 9--10 são executadas em $\Theta(F)$ e as restantes linhas são executadas em $\Theta(1)$.

Conclui-se então que a complexidade temporal do programa é majorada pela função \textbf{get-result}, logo a complexidade total é \O{$F(V+E)$\log{V}}.
\\
\\
Analisando a complexidade espacial, nota-se que é \O{V+E+F}, uma vez que para além do grafo (e do seu transposto calculado na função \textbf{get-result}) apenas são necessários vectores com V ou F elementos (mais concretamente, o vetor de filiais inicializado na linha 2 da função principal, e os vetores das linhas 3 e 4 da função \textbf{get-result}).

\section{Avaliação Experimental dos Resultados}

Fazendo dois tipos de testes diferentes, um em que varia o número de filiais e outro onde varia o número de vértices e arestas, é possível ver que o primeiro tem um crescimento linear, e o segundo tem o crescimento do tipo $y = x\log{x}$.

\begin{center}
\begin{tikzpicture}
		\begin{axis}[
				title=Relação Tempo-Dimensão,
                xlabel={Dimensão (F)},
				ylabel={Tempo (s)}
		]
		\addplot table {branches_results.dat};
		\end{axis}
\end{tikzpicture}
\end{center}

\begin{center}
\begin{tikzpicture}
		\begin{axis}[
				title=Relação Tempo-Dimensão,
                xlabel={Dimensão (V+E)},
				ylabel={Tempo (s)}
		]
        \addplot table {test_results.dat};
		\end{axis}
\end{tikzpicture}
\end{center}

\begin{thebibliography}{9}
		\bibitem{algs3ed}
				Thomas H. Cormen,
				Charles E. Leiserson,
				Ronald L. Rivest,
				Clifford Stein,
				\emph{Introduction to Algorithms},
				3rd Edition,
				September 2009
\end{thebibliography}

\end{document}
